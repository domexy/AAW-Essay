\documentclass[12pt]{scrartcl}

\title{Quality over quantity, a better goal for scientific progress}
\author{Kai Domhardt}
\date{\today}

\begin{document}

\maketitle

\paragraph{Introduction}
All life forms, be it plant or animal, share one common goal, survival. Man is not different. The fear of our own mortality is a central motivator in our past, current and future actions. For the longest time death occured mostly through external factors, starvation, sickness and war. Due to the hightend quality of life, we have reached in the past century, the prevalent cause of death in the developed world has shifted from external to internal factors. In the third millenium the driving force behind survival is no longer a political or economic one. The delaying of death is now only achievable through scientific progress, but this can bring unforseen social and economic consequences.


\paragraph{Con 1: social}
People, who have lived a long life have rich pool of experiences to draw from, when it comes to making decisions and understanding the world around them.
\textbf{Refutation: age gap/devide ; memory horizont for our own experience}
This thought might lead one to insinuate that seniors take in more [stuff] to reach their conclusion, but this would neglect the existence of cognitive biases.


\paragraph{Con 2: economic}
Some people think, that the additional lifetime equates to additional productivity. 
\textbf{Refutation:}
This idea has been propagated by many futuristic utopian fictional works. But as contemporary studies show, the plausability of this idea is the same as the medium it stems from: fiction. During the last decades Germany has been experiencing what has come to be known as "greying of society". A decreasing population, due to lack of births, resulting in an upwards shift of the mean age. Today a smaller percentage of the population has to provide for the growing number of senior citizens that have retired from the working life. Even though this puts a strain on the economy, increasing the age of retirement has been fought against and reject time and time again.

\paragraph{Conclusion:}
Considering these factors, lead to the conclusion that extended life leads to many new problems we as community have to overcome.
[Solution = Quality of life]
Our scientify effort must be to strive quality of life not quantity in life.
\end{document}
